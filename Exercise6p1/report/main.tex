% !TeX spellcheck = en_GB
\documentclass[10pt,respuestas,a4]{aleph-examen}
%\documentclass[10pt,a4]{aleph-examen}
% Intercambiar el comentario de las primeras lineas
% para mostrar/ocultar respuestas

% Se recomienda leer la documentación de esta
% clase en https://github.com/alephsub0/LaTeX_aleph-examen

% -- Paquetes
%\usepackage{aleph-comandos}
\usepackage{physics}
\usepackage{hyperref}
\usepackage[capitalise]{cleveref}

% -- Datos del examen
\institucion{Exercise 6.1 Regularization}
% \carrera{}
\asignatura{Applied Deep Learning in Physics and Engineering (1FA370)}
\tema{Mastering Model Building}
\autor{Damián López Hermida \& Adrián Silva Caballero}
\fecha{Autumn 2025}

\logouno[2.5cm]{Logos/uu_logo}
\xdefinecolor{colortext}{RGB}{151,31,48}
\definecolor{colordef}{HTML}{0030A1}
\fuente{montserrat}

% -- Comandos extra
% \geometry{margin=2cm} % - Para cambiar los márgenes

%%%%%%%%%%%%%%%%%%%%%%%%%%%%%%%%%%%%%%%%
%%%%%%%%%%%%%%%%%%%%%%%%%%%%%%%%%%%%%%%%
%%%%%%%%%%%%%%%%%%%%%%%%%%%%%%%%%%%%%%%%
\begin{document}

\encabezado

Open the Tensorflow Playground (\href{https://playground.tensorflow.org/#activation=tanh&batchSize=10&dataset=xor&regDataset=reg-plane&learningRate=0.03&regularizationRate=0&noise=50&networkShape=4,2&seed=0.77633&showTestData=false&discretize=false&percTrainData=50&x=true&y=true&xTimesY=false&xSquared=false&ySquared=false&cosX=false&sinX=false&cosY=false&sinY=false&collectStats=false&problem=classification&initZero=false&hideText=false}{playground.tensorflow.org}) and select on the left the checkerboard pattern as the data basis. In features, select the two independent variables x1 and x2 and set noise to 50\%. (if you follow the link, the settings should have been applied automatically). Start the network training and describe your observation when doing the following steps:

\begin{preguntas}
%%%%%%%%%%%%%%%%%%%%%%%%%%%%%%%%%%%%%%%%
%%%%%%%%%%%%%%%%%%%%%%%%%%%%%%%%%%%%%%%%
%%%%%%%%%%%%%%%%%%%%%%%%%%%%%%%%%%%%%%%%
\item \label{Q1}
   Choose a deep and wide network and run for > 1000 epochs.

\begin{respuesta}
	
    
    
\end{respuesta}

%%%%%%%%%%%%%%%%%%%%%%%%%%%%%%%%%%%%%%%%
%%%%%%%%%%%%%%%%%%%%%%%%%%%%%%%%%%%%%%%%
%%%%%%%%%%%%%%%%%%%%%%%%%%%%%%%%%%%%%%%%
\item \label{Q2}
Apply L2 regularization to reduce overfitting. Try low and high regularization rates.

\begin{respuesta}
	
	
\end{respuesta}

%%%%%%%%%%%%%%%%%%%%%%%%%%%%%%%%%%%%%%%%
%%%%%%%%%%%%%%%%%%%%%%%%%%%%%%%%%%%%%%%%
%%%%%%%%%%%%%%%%%%%%%%%%%%%%%%%%%%%%%%%%
\item \label{Q3}
Compare the effects of L1 and L2 regularizations.

\begin{respuesta}
	

\end{respuesta}


\end{preguntas}

\end{document}
